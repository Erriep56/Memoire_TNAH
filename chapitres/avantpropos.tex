\lettrine{C}e mémoire s’inscrit dans le cadre d’un stage mené conjointement à l’Institut national d'histoire de l'art et au sein du consortium Huma-Num pictorIA, au cours de ma seconde année de Master 2 « Technologies numériques appliquées à l’histoire » à l’École nationale des Chartes. L’objectif principal de ce stage était l’indexation iconographiques de peintures issues du RETIF (Répertoire des tableaux italiens dans les collections publiques françaises) à l’aide d’outils d’intelligence artificielle. Ce projet, intitulé \textit{Iconographia}, s’est déroulé sur une période de quatre mois, du 1\textsuperscript{er} avril au 31 juillet 2025, sous la direction de Federico Nurra et de Julien Schuh. J’ai été chargé de cette mission en raison de ma double formation en histoire de l’art et en technologies numériques. Diplômé du deuxième cycle de l’École du Louvre en 2023, j'avais consacré mes recherches à la gravure d’interprétation en Italie à l’Époque moderne, ce qui a nourri mon intérêt pour l’histoire de la peinture italienne et enrichi mes connaissances dans ce domaine. Au cours de ce projet, j’ai abordé la question principalement sous l’angle de l’historien de l’art, étant la seule personne à travailler à temps plein pour le mener. Il convient toutefois de préciser que, malgré mon intérêt pour les technologies d’intelligence artificielle, ma maîtrise des modèles d’apprentissage profond et de leurs usages reste partielle. Cette limite a entraîné certaines difficultés à manipuler ces technologies, que je présenterai au fil des chapitres. Loin de constituer un simple frein, elles ont ouvert de nouvelles perspectives et permis d’apporter un regard neuf aux pratiques d’indexation iconographique.

Je tiens tout d’abord à exprimer ma gratitude à ma directrice de mémoire, Ségolène Albouy, qui m’a accompagné tout au long de sa rédaction, et à Gennaro Toscano de présider mon jury de soutenance. Je remercie chaleureusement Federico Nurra et Julien Schuh pour leur encadrement au cours de ce stage, ainsi que mes collègues à l’INHA, Pierre-Yves Laborde, Jean-Christophe Carius, Margaux Faure, Maria Francesca Bocchi et Natacha Grim pour leurs conseils et leur soutien. Ma reconnaissance va également à mes collègues stagiaires et amis Camille Samsa, Fantin Le Ber et Mathieu Taybi, ainsi qu’à l’ensemble de mes camarades de master, pour les échanges stimulants que nous avons partagés. Je remercie également Marion Charpier et Emmanuelle Bermès pour leur accompagnement. J’adresse un merci particulier à Teoman Akgönül et Agathe Ménétrier, pour leur disponibilité et leur écoute, ainsi qu’à Johanna Daniel, dont les réflexions ont enrichi les miennes. Je voudrais témoigner ma gratitude envers Hans Brandhorst et Etienne Posthumus d'avoir accepté de s'entretenir avec moi au début de mes recherches. Enfin, je souhaite remercier mes amis et ma famille pour leur présence et leur soutien constant : Eléa Dargelos, Solveig d’Aboville, Grégoire Hignard, Thaïs Wsevoloski, Mathieu Darrieutort, Théophile Piquemal-Tabou et Inès Hennequin. Pour leur aide dans la relecture de ce mémoire, j'adresse un merci particulier à ma maman et à Adèle Bugaut. 
