Comme nous l'évoquions précédemment, les méthodes d'indexation sont intimement liées aux vocabulaires qu’elles utilisent. En histoire de l'art, la plupart des vocabulaires ont vu leur naissance dans la deuxième moitié du XX\textsuperscript{e} siècle, et surtout dans les années 1980, avec le développement des systèmes informatiques dans les institutions patrimoniales et l’amélioration des pratiques de catalogage. Des vocabulaires d’indexation iconographique génériques ont alors vu le jour, destinés à décrire l’ensemble des représentations artistiques, tandis que des vocabulaires spécialisés sont employés pour un domaine d’étude précis ou un projet particulier. La question de l'alignement de ces vocabulaires est un enjeu crucial pour l'interopérabilité des bases de données et la facilitation de la recherche iconographique.

\section{Vocabulaires génériques et spécifiques}

Différents types de vocabulaires peuvent être mobilisés pour décrire les représentations présentes au sein des objets patrimoniaux, selon l’ampleur de l’analyse envisagée et le degré de spécialisation des approches adoptées. Certains sont spécifiquement liés à des domaines d’étude particuliers, tandis que d’autres, plus généralistes, visent à proposer une description uniforme des représentations, quel que soit leur contexte de création.
Ainsi, les vocabulaires spécialisés sont souvent associés à des projets de recherche ou à des disciplines précises. Pour les objets archéologiques, par exemple, le Lexicon Iconographicum Mythologiae Classicae (LIMC) constitue une référence. Publié en huit tomes doubles entre 1981 et 2009, il est consacré aux représentations figurées issues des mythologies grecque, étrusque et romaine, appliquées à divers objets antiques (céramique, sculpture, orfèvrerie, glyptique, peinture, mosaïque, etc.)\cite{fondationinternationalepourlelimcLexiconIconographicumMythologiae1981, fondationinternationalepourlelimcLexiconIconographicumMythologiae2009}. 

Dans le domaine de l’art médiéval, plusieurs répertoires et dictionnaires iconographiques ont vu le jour, parmi lesquels l’Index of Medieval Art de l’université de Princeton. Fondé en 1917 par le professeur Charles Rufus Morey, il a d’abord été conçu au format papier, rassemblant un index des iconographies chrétiennes, puis juives, islamiques et non religieuses, accompagné de photographies. Informatisé en 1991, il est accessible en ligne depuis 2017\footcite{HistoryIndex}. Plus récemment, des projets tels que l’Ontologie du christianisme médiéval en images, mise en ligne en 2023 par l’Institut national d’histoire de l’art, proposent également un lexique contrôlé pour les représentations figurées dans le contexte médiéval\footcite{beaudOntologieChristianismeMedieval2023}.
Pour les œuvres d’art picturales, il existe certes des dictionnaires spécialisés, mais ce sont surtout des systèmes iconographiques généralistes qui sont employés pour leur description. Certains ont été développés dans le cadre d’institutions, comme le système d’indexation par mots-clés du Warburg Institute\footcite{thewarburginstituteSummaryGuidePhotographic1988}, ou encore le Thesaurus for Graphic Materials, publié par la Library of Congress et principalement utilisé pour les estampes et les photographies\footnote{Aujourd’hui disponible en ligne : \cite{libraryofcongressThesaurusGraphicMaterials2024}}.

Aujourd’hui, les vocabulaires iconographiques les plus utilisés dans les institutions, notamment pour la description des œuvres d’époque moderne et contemporaine, sont \textit{ICONCLASS}, les thésaurus développés par le Getty Research Institute, et, en France, le thésaurus Garnier. Parallèlement, Wikidata prend une place croissante dans la description des représentations, grâce à l’implication grandissante de contributeurs et d’institutions qui y participent activement.

Le système iconographique Iconclass a été conçu à l’initiative de l’historien de l’art néerlandais Henri van de Waal. Publié une première fois en 1968 sous l’acronyme D.I.A.L. (textit{Decimal Index of the Art of the Low Countries})\footcite{vandewaalDecimalIndexArt1968}, il prend rapidement le nom d’Iconclass, contraction d’ICONographic CLASSification (parfois rendu ICONographic CLASsification System). Initialement prévu pour paraître en volumes imprimés, il fait l’objet d’une édition augmentée et révisée entre 1973 et 1985, comprenant sept volumes consacrés au système, sept volumes de bibliographie et trois volumes d’index\footcite{vandewaalIconclassIconographicClassification1974}. Dès le début des années 1970, Van de Waal anticipe un usage informatisé du dispositif, ce qui conduit à une collaboration avec le Centraal Rekeninstituut de Leyde et au portage d’Iconclass sur un ordinateur IBM mainframe en 1978\footcite[p. 202]{brandhorstICONCLASSKeyCollaboration2017}. Après des versions diffusées sur CD-ROM dans les années 1990, le système est finalement mis en ligne entre 1999 et 2000\footcite{brandhorstSurveyVersionsIconclass}. Le principe d’Iconclass repose sur une classification hiérarchique des représentations, organisée en dix classes principales, chacune subdivisée en catégories plus fines, couvrant notamment les domaines de la Nature, de L’Être Humain, de l’Histoire, de la Bible ou des Mythologies. Chaque concept est identifié par un code alphanumérique, dont la longueur indique le degré de précision : plus le code est détaillé, plus la description est fine.

Parallèlement aux travaux de Van de Waal, des initiatives de normalisation des vocabulaires émergent aux États-Unis, portées par le Getty Trust, puis par le Getty Research Institute (GRI), fondé en 1985. Dès la fin des années 1970, est engagé le développement de l’Art \& Architecture Thesaurus (AAT), destiné à répondre aux besoins des bibliothèques et des revues spécialisées en matière de catalogage et d’indexation\footcite{gettyresearchinstituteAAT2024}. Le projet est initié par les directeurs de bibliothèques et des historiens de l’art Toni Petersen, Dora Crouch et Pat Molholt,  avec le soutien financier du Getty Trust. Structuré au cours des années 1980 par le Getty Vocabulary Program, l’AAT paraît en deux éditions imprimées, en 1990 et 1994\footcite{petersenArtArchitectureThesaurus1990}. De portée générale, l’AAT rassemble des termes multilingues pour désigner des agents, des types d’œuvres, des rôles, des matériaux, des styles, des cultures et des techniques de création. Si certains de ces termes concernent aussi les sujets iconographiques, c’est surtout la Getty Iconography Authority (IA), intégrée au programme des vocabulaires du GRI, qui s’attache directement aux représentations iconographiques. Plus spécialisé, ce vocabulaire recense des noms propres, des personnages religieux ou fictifs, des événements historiques, ainsi que des titres d’œuvres littéraires ou de spectacles, en intégrant des relations et des repères chronologiques. L’accent est particulièrement mis sur les cultures non occidentales,  notamment asiatiques et américaines, tout en établissant des références croisées avec d’autres ressources, telles qu’Iconclass pour le contexte occidental\footcite{gettyresearchinstituteCONAIA2024}. Afin de faciliter les mises à jour régulières, une version numérique de l’AAT est mise en ligne en 1997, accompagnée d’une interface de recherche. Depuis, l’ensemble des vocabulaires Getty, dont l’AAT et l’IA, est accessible via le site du Getty Research Institute\footcite{gettyresearchinstituteGettyVocabularies}.

En France, s’est développé à la même époque le \textit{Thésaurus iconographique, système descriptif des représentations de François Garnier}, plus couramment appelé thésaurus Garnier. Initié en 1976 à la demande du Ministère de la Culture, il répondait aux besoins de l’Inventaire général des monuments et des richesses artistiques de la France ainsi qu’à ceux de la Direction des musées de France\footcite[p. 11]{garnierThesaurusIconographiqueSysteme1984}. Son objectif était de fournir une liste de descripteurs destinés à l’indexation des documents iconographiques et à l’alimentation des bases de données créées par ces services. Conçu avec une vocation universelle, ce thésaurus visait à offrir un langage normalisé permettant de décrire toute représentation, indépendamment de sa date, de sa nature ou de son support. Le thésaurus Garnier demeure aujourd’hui « l'outil retenu par le ministère de la Culture pour l'analyse des représentations figurées dans les bases de données patrimoniales » en France\footnote{\cite{ministeredelacultureThesaurusIconographiqueSysteme2020}}. La première version a été publiée sous forme d’ouvrage aux éditions Léopard d’or. Richement illustrée de reproductions de gravures sur bois accompagnées d’exemples d’indexation conforme aux règles du thésaurus, offrant ainsi aux utilisateurs des cas pratiques d’application\footcite{garnierThesaurusIconographiqueSysteme1984}. Cette version initiale se composait de quatre rubriques : une liste hiérarchisée de termes pour décrire une représentation, une liste non exhaustive de noms propres ou de termes particuliers (par exemple des lieux), les sources textuelles dont les représentations sont issues et des termes permettant de dater le contenu représenté (et non l’exécution de l’œuvre). De nombreux musées français utilisent encore cette version imprimée, désormais mise à disposition en ligne grâce à une numérisation complète réalisée par le ministère de la Culture\footcite{ministeredelacultureThesaurusIconographiqueSysteme2020}. Celui-ci propose également une version hébergée sur Opentheso, structurée en 22 catégories et comprenant 32 895 concepts\footcite{ministeredelacultureThesaurusIconographiqueGarnier}. L’Institut national d’histoire de l’art (INHA) diffuse quant à lui une version distincte, totalisant 16 362 concepts répartis en 24 catégories\footcite{institutnationaldhistoiredelartThesaurusAGORHAGarnier2021}. Enfin, le thésaurus iconographique utilisé par Mandragore, la base de données recensant les enluminures des manuscrits de la Bibliothèque nationale de France, correspond à une version dérivée et simplifiée du thésaurus Garnier\footcite{bibliothequenationaledefranceThesaurusIconographiqueMandragore2024}.

Enfin, Wikidata constitue aujourd’hui une alternative pour décrire les représentations au moyen de ses propriétés\footcite{Wikidata}. Créée en 2012 par la Wikimedia Foundation, cette base de connaissances libre et collaborative est accessible en ligne\footcite{perezWikipediasNextBig2012}. Elle fonctionne comme un référentiel sémantique où chaque entité est décrite par un élément (item), identifié par un identifiant unique. Wikidata est utilisée pour structurer et centraliser les données des projets Wikimedia (comme Wikimedia Commons ou Wikipedia), mais aussi par des institutions patrimoniales et scientifiques afin d’enrichir ou d’interconnecter leurs bases de données. Chaque élément y est défini par un label, c’est-dire le nom principal, complété d’une description et éventuellement d’alias correspondant à des variantes ou synonymes. Tous ces champs sont multilingues. Les images et les éléments peuvent être reliés entre eux grâce aux propriétés de Wikidata. Par exemple, dans le cas de la description iconographique, la propriété P180 (dépeint / depicts) permet d’associer une image à l’élément qu’elle représente\footcite{P180DepeintDepicts}.

\section{Systèmes descriptifs et systèmes d'indexation}

En 1994, Roelof van Straten faisait la distinction entre des “systèmes descriptifs” (\textit{descriptive systems}) qui utilisent uniquement du langage naturel et les “systèmes d’indexation” (\textit{indexing systems}) dont le but n’est pas de décrire les images mais de les indexer en leur attribuant un ou plusieurs codes ou mot-clés\footcite[p. 101]{vanstratenIconographyIndexingIconclass1994}. Cette distinction permet de comprendre l’utilisation des différents systèmes et vocabulaires utilisés dans la description iconographique, surtout dans le cadre de projets numériques. Avec l’apparition de l’informatisation, on observe une mutation des systèmes descriptifs en systèmes d’indexation en s’appuyant sur les technologies fournies par les outils numériques.

Par système descriptif, on entend des systèmes reposant sur le langage naturel, utilisant des mots ou des expressions pour décrire une image. Les thésaurus entrent dans cette catégorie, car ils fournissent des termes contrôlés permettant de qualifier les représentations. Le thésaurus iconographique Garnier a été conçu dans cet esprit. Dans sa version papier, il propose ainsi une liste hiérarchisée de plus de 3 200 descripteurs, répartis en 23 classes. Parmi celles-ci, 14 classes sont considérées comme fixes, tandis que les autres peuvent être adaptées à des collections particulières\footcite[p. 15]{garnierThesaurusIconographiqueSysteme1984}. Ces classes se répartissent en deux catégories principales : thèmes et sujets. Les thèmes regroupent des motifs communs, classés en 14 catégories, telles que la nature, les transports et la communication, ou les ornements. Les sujets se réfèrent à des scènes ou des personnages identifiables, issus de textes bibliques, de la mythologie, de l’histoire ou de récits. Pour représenter les relations hiérarchiques entre les éléments et les catégories, des parenthèses sont ajoutées dans les descriptions. Par exemple, pour la peinture représentant Saint Apolline et Saint Sébastien du Pérugin, conservée au Musée de Grenoble, la description selon le thésaurus papier devait être formulée ainsi :

\begin{quote}
Groupe de figures (sainte Apolline, pince, dent, en pied, de face, Saint Sébastien, flèche).
\end{quote}

Pour la peinture de \textit{Persée délivrant Andromède} de Paul Véronèse (1528-1588) conservée au Musée des beaux-arts de Rennes (inv. 1801.1.1), la description iconographique pourrait être formulée ainsi :

\begin{quote}
Scène mythologique (Persée, épée, Andromède, monstre : marin, bord de mer, délivrance, rocher, chaîne) - fonds de paysage (ville)
\end{quote}

Ainsi, la description formulée s’apparente à une description formulée en langage naturel, en s’appuyant sur les termes fournis par le thésaurus. 

Quant aux systèmes d’indexation, ils n’avaient pas pour objectif de fournir une description complète des images, mais plutôt d’attribuer des mots-clés ou des codes normalisés permettant de les classer et les retrouver dans une base de données. Ils reposent sur des concepts pouvant désigner une idée, un terme, une scène, un motif ou un élément de représentation, chacun étant identifié par un code ou un mot-clé. Iconclass en constitue l’exemple le plus abouti : il utilise des codes alphanumériques qui restituent la hiérarchie du système et précisent progressivement un concept. Le premier caractère du code correspond à la catégorie la plus générale, puis les suivants affinent la description. Ainsi, le \textit{Persée délivrant Andromède} de Paul Véronèse est indexé par le code 94P211. Celui-ci indique successivement : la mythologie classique (9), les légendes héroïques grecques (94), l’histoire de Persée (P), ses amours (2) avec Andromède (1), et plus précisément l’épisode où Persée tue le monstre marin pour la délivrer (1). La notice correspondante décrit en français : « 94P211 Persée tue le monstre marin (ou le pétrifie en lui montrant la tête de Méduse) et délivre Andromède ; il se peut que les parents de celle-ci, Céphée et Cassiopée, observent la scène »\footnote{ICONCLASS 94P211, \url{https://iconclass.org/94P211}}. Dans d’autres cas, le code peut inclure un nom propre, comme pour saint Sébastien (11H(SEBASTIAN))\footnote{ICONCLASS 11H(SEBASTIAN),  \url{https://iconclass.org/fr/11H\%28SEBASTIAN\%29}} ou sainte Apolline (11HH(APOLLONIA))\footnote{ICONCLASS 11HH(APOLLONIA), \url{https://iconclass.org/11HH\%28APOLLONIA\%29}}. Des qualificatifs supplémentaires permettent également de préciser une action ou un contexte : par exemple, un (+1) à la suite d’un code désignant un saint signale la présence d’un ange à ses côtés, tandis que 34B12(+946) désigne un chat dormant (34B12 pour le chat, (+946) pour l’action de dormir)\footcite[p. 59]{vanstratenIconographyIndexingIconclass1994}. L’usage des systèmes d’indexation comme Iconclass se distingue donc nettement d’un thésaurus. Ils sont plus complexes à utiliser, mais sont capables de cerner le sujet principal ou un motif précis tout en conservant la hiérarchie des idées associées.

\section{Vers une interopérabilité des outils d’indexation}

Avec la mise en ligne des thésaurus et systèmes de description iconographique, on observe une mutation des “systèmes descriptifs” à des “système d’indexation”. Dans ce cadre, chaque descripteur devient un concept doté d’un identifiant, permettant le plus souvent de former l’URL du concept. Ce concept contient toujours un libellé principal (label), parfois enrichi de variantes, de traductions, ou d’une description. C’est le cas des vocabulaires du Getty Research Institute, mais aussi du thésaurus Garnier dans ses instances gérées par le Ministère de la Culture et par l’Institut national d’histoire de l’art. La majorité de ces vocabulaires contrôlés reposent sur le format RDF (Resource Description Framework), standard du W3C pour représenter des données structurées et interopérables sur le web\footcite{rdfworkinggroupResourceDescriptionFramework2014}. Ce format exprime les informations sous forme de triplets : un sujet (par ex. un concept du thésaurus), un prédicat (par ex. « a pour libellé ») et un objet (par ex. « La Crucifixion »). Pour uniformiser ces représentations, on utilise souvent le modèle SKOS (Simple Knowledge Organization System)\footcite{milesSKOSSimpleKnowledge2009}, qui définit les unités du thésaurus comme des <skos:Concept>, leur libellé de référence avec <skos:prefLabel>, et les relations hiérarchiques avec <skos:broader> pour le concept plus large, soit le parent dans la hiérarchie, et <skos:narrower> pour le concept plus spécifique, soit l’enfant dans la hiérarchie. Cette approche favorise l’interopérabilité, le multilinguisme et la navigation hiérarchique dans les vocabulaires.
Cependant, une véritable interconnexion entre vocabulaires reste rare. C’est l’ambition de Wikidata, qui relie ses entités à des référentiels externes via des propriétés dédiées. Ainsi, environ 1 109 concepts du Getty Iconography Authority\footnote{Calculé à partir de la requête SPARQL suivante : \url{https://query.wikidata.org/sparql?query=SELECT\%20(COUNT(DISTINCT\%20\%3FId)\%20AS\%20\%3Fcount)\%20WHERE\%20\%7B\%20\%3Fitem\%20wdt\%3AP5986\%20\%3FId.\%20\%7D} (consultée le 3/09/2025)}, 24 597 du Getty Art and Architecture Thesaurus\footnote{Calculé à partir de la requête SPARQL suivante : \url{https://query.wikidata.org/sparql?query=SELECT\%20(COUNT(DISTINCT\%20\%3FId)\%20AS\%20\%3Fcount)\%20WHERE\%20\%7B\%20\%3Fitem\%20wdt\%3AP1014\%20\%3FId.\%20\%7D} (consultée le 3/09/2025)},  4 131 identifiants Iconclass\footnote{Calculé à partir de la requête SPARQL suivante : \url{https://query.wikidata.org/sparql?query=SELECT\%20(COUNT(DISTINCT\%20\%3FId)\%20AS\%20\%3Fcount)\%20WHERE\%20\%7B\%20\%3Fitem\%20wdt\%3AP1256\%20\%3FId.\%20\%7D} (consultée le 3/09/2025)} y sont déjà utilisés. Notons également que 257 entités Wikidata sont associées à la fois à un concept du Getty Iconography Authority et à un concept d’ICONCLASS\footnote{Calculé à partir de la requête SPARQL suivante : \url{https://query.wikidata.org/sparql?query=SELECT\%20(COUNT(DISTINCT\%20\%3Fitem)\%20AS\%20\%3Fcount)\%20WHERE\%20\%7B\%0A\%20\%20\%3Fitem\%20wdt\%3AP5986\%20\%3FiaId\%3B\%0A\%20\%20\%20\%20wdt\%3AP1256\%20\%3FiconclassId.\%0A\%7D} (consultée le 03/09/2025)}. En revanche, aucune propriété n’existe encore pour le thésaurus Garnier, qu’il s’agisse de la version du ministère de la Culture ou de l’INHA. La création d’une telle propriété et l’alignement des descripteurs aux entités Wikidata pourraient permettre de désenclaver le thésaurus Garnier, d’assurer sa compatibilité multilingue et de relier les bases de données iconographiques françaises à celles internationales.
Un obstacle majeur à prendre en compte pour l'interopérabilité des systèmes de description iconographique reste la différence de granularité entre eux. Iconclass décrit chaque concept par un identifiant unique, tel que 94P211 pour Persée délivrant Andromède, tandis que le thésaurus Garnier nécessiterait de combiner plusieurs termes (« Scène mythologique », « Persée », « Andromède », « délivrance ») pour exprimer la même idée. Malgré cela, Iconclass mène une politique active de liaisons vers d’autres bases et références bibliographiques, même si l’accès à ses services avancés (Iconclass+) reste payant\footcite{brandhorstVirtualLibraryIconographic2023}.

La diversité des vocabulaires contrôlés et des systèmes d’indexation représente à la fois un potentiel majeur pour l’indexation iconographique et un défi pour son efficacité : il reste nécessaire de mettre ces outils en correspondance afin de permettre le croisement des descriptions. Chacun d’eux contribue cependant à décrire les images avec précision, facilitant l’exploration des bases de données et la compréhension des représentations.
