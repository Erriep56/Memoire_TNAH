\section{L'indexation iconographique en histoire de l’art}

Bien que la pratique de l’indexation remonte à plusieurs siècles,ce n’est qu’à partir du milieu du XX\textsuperscript{e} siècle qu’elle a fait l’objet d’une véritable théorisation, notamment pour la description du contenu intellectuel des livres et des documents conservés dans les bibliothèques\footnote{Consulter à ce propos : \cite{amarFondementsTheoriquesLindexation2000}.}. Lorsqu’elle s’applique aux objets muséaux, cette méthode reprend les principes développés pour les textes, tout en les enrichissant par les apports de l’archéologie et de l’histoire de l’art, indispensables à leur compréhension et à leur interprétation.

Pour définir et expliciter la pratique de l’indexation dans la pratique documentaire du catalogage, l'Association française de normalisation (AFNOR) la présente ainsi : l’indexation est « l’opération qui consiste à décrire et à caractériser un document à l’aide de représentations des concepts contenus dans ce document, c’est-à-dire à transcrire en langage documentaire les concepts après les avoir extraits du document par une analyse. La transcription en langage documentaire se fait grâce à des outils d’indexation tels que thésaurus, classifications, etc. […] La finalité de l’indexation est de permettre une recherche efficace des informations contenues dans un fonds de documents et d’indiquer rapidement, sous une forme concise, la teneur d’un document \footcite[norme NF Z47-102, vol. 1, p. 286]{associationfrancaisedenormalisationDocumentationRecueilNormes1993} ». L’indexation iconographique, en histoire de l’art, constitue une application spécifique de ce principe : elle consiste à décrire le contenu des images à l’aide de descripteurs afin de faciliter l’accès aux œuvres, sans cependant dispenser sa consultation\footnote{« Le rôle du système de recherche documentaire est de faciliter l’accès au document et non de dispenser sa consultation » \cite[p. 15]{garnierThesaurusIconographiqueSysteme1984}.}. Avec le développement des bases de données, ces pratiques d’indexation reposent désormais presque systématiquement sur des vocabulaires contrôlés pour décrire les documents. Pour les œuvres d'art, où peuvent être présentés des objets, des personnages, mais aussi des concepts ou des scènes souvent codifiées, les thésaurus jouent un rôle central pour décrire ces éléments de manière normalisée. Ainsi, les méthodes et manuels d’indexation sont fréquemment associés aux thésaurus qu’ils accompagnent, afin de garantir une utilisation la plus uniforme possible du vocabulaire et d'assurer l’efficacité des recherches qui en découlent. Les principaux systèmes d’indexation iconographique, tels qu’\textit{Iconclass} et le thésaurus Garnier, sont chacun accompagnés d’études et de guides destinés à encadrer leur utilisation, fournissant également des réflexions sur cette pratique documentaire\footnote{Nous décrirons plus en détail l'histoire, l'utilisation et les particularités de ces thésaurus dans le chapitre 2.}.

Cependant, les recommandations concernant l’usage des thésaurus ne fournissent pas une méthode claire et rigoureuse pour l’indexation des images, et ce pour plusieurs raisons. La première est le manque d’études consacrées à cette pratique en histoire de l'art. Certaines études comme celles de Sarah Shatford prennent en considération toutes les images et restent trop génériques pour des œuvres picturales\footcites{shatfordDescribingPictureThousand1984}{shatfordAnalyzingSubjectPicture1986}. Dans son manuel sur l’usage du système iconographique \textit{Iconclass} publié en 1994, l'historien d'art Roelof van Straten remarque que la communauté scientifique s’est rarement penchée sur des questions pourtant essentielles liées à l’indexation, telles que la nature même de la tâche de l’indexeur ou la profondeur à laquelle l’indexation doit être effectuée.\footnote{\cite{vanstratenIconographyIndexingIconclass1994}, p. 37 : “It is astonishing that even the most basic questions regarding iconographic indexing, such as “What is the real task of the iconographic indexer?” or “How far ‘in depth’ should we index ?” do not seem to have been discussed among art historians.”}. Une autre raison est celle de la grande variété des images à indexer présente dans les institutions patrimoniales ou les collections, rendant difficile l’écriture d’une méthode efficace. François Garnier, dans l'introduction de son thésaurus iconographique utilisé par les musées français, indique alors « qu'aucune grille d'analyse ne s'impose comme une nécessité »\footcite[p. 25]{garnierThesaurusIconographiqueSysteme1984}. 

S’il n’existe pas de consensus sur la manière de rédiger une indexation « parfaite » pour décrire une image, ces auteurs soulignent que l’indexeur doit avant tout se laisser orienter par la finalité de sa tâche : permettre un accès efficace aux images au sein d’une base de données et faciliter les recherches, présentes comme futures, des analystes d’images. Ils proposent à cet effet une liste d’éléments essentiels à prendre en compte pour décrire une représentation\footnote{\cite[p. 57-58]{vanstratenIconographyIndexingIconclass1994}, \cite[p. 25]{garnierThesaurusIconographiqueSysteme1984}}. Cette description doit d’abord rendre compte de la signification principale de l’image et de son contenu, en identifiant la scène, les personnages et les lieux représentés. Elle doit ensuite être enrichie par le relevé des motifs secondaires, liés ou non à l’action principale. Une attention particulière doit être portée aux éléments inhabituels, surtout lorsqu’ils présentent un intérêt historique, ainsi qu’aux détails que l’image met en avant. L’objectif est surtout de restituer autant de concepts du contenu visuel que possible, même s’il existe toujours une part inévitable de subjectivité dans le travail d’indexation. Sara Shatford insiste également sur le besoin d’être concis pour éviter de fausser les recherches dans les bases de données et d’abord privilégier les idées les plus importantes, en fonction du public visé\footcite{shatfordDescribingPictureThousand1984}. La qualité de l’indexation dépend également de plusieurs paramètres : la qualité du système d’indexation utilisée et sa compréhension par l’indexeur, l’expérience et les connaissances de ce dernier, et la lisibilité de l’image\footnote{\cite[p. 24]{garnierThesaurusIconographiqueSysteme1984}, \cite[p. 37]{vanstratenIconographyIndexingIconclass1994}}.

\section{Sujets, genre, motifs et éléments}

L’usage du vocabulaire générique que nous emploierons tout au long de ce mémoire pour décrire les images ne fait pas l’objet d’un accord unanime chez les historiens de l’art\footnote{Nous avons consulté à ce propos différents ouvrages généralistes qui abordent le vocabulaire : \cite{vanstratenIntroductionIconography1994}, \cite{barralialtetDictionnaireCritiqueDiconographie2003}, \cite{bergeonlanglePeintureDessinVocabulaire2009}, \cite{gervereauVoirComprendreAnalyser2020}.}. Les termes « motifs », « thèmes », « sujet », « genre » et « éléments » ne sont pas toujours employés dans le même sens en fonction des publications. C’est pourquoi nous exposons ici la manière dont nous utiliserons ces termes. 

Le terme « sujet » désigne le thème représenté dans une image ou une œuvre d’art, c’est-à-dire l’objet principal de la représentation. Il peut se référer au sujet général, correspondant à la catégorie de la représentation (portrait, paysage, scène religieuse ou mythologique), ou à une scène spécifique, c’est-à-dire une situation particulière représentée. Le sujet sert souvent à formuler le titre d’une œuvre lorsqu’aucun titre n’a été donné par l’artiste ou forgé par l’historiographie. Dans ce cas, nous préférons utiliser le terme « scène » pour plus de précision.

Dans l’histoire de la peinture, le sujet général, également appelé « genre », relève d’une classification théorisée dès le XVIIᵉ siècle. Dans son introduction aux conférences de l’Académie royale de peinture en 1667, André Félibien définit plusieurs genres iconographiques selon une hiérarchie\footnote{felibienConferencesLAcademieRoyale1667} : la peinture d’histoire, qui occupe le premier rang, regroupe les œuvres représentant des événements historiques ou illustrant des textes, tels que la Bible ou la mythologie grecque et romaine. Le portrait constitue le deuxième genre pictural, suivi de la scène de genre, qui représente des scènes anecdotiques ou familières, souvent inspirées du quotidien. Viennent ensuite les genres qui mettent moins l’accent sur la figure humaine : le paysage, puis la nature morte, centrée sur la représentation d’éléments inanimés. Cette classification a profondément marqué l’histoire picturale, en particulier à l’Époque moderne. Si elle a désormais perdu son caractère hiérarchique, elle reste prégnante aujourd’hui dans les thésaurus comme les dictionnaires, avec quelques adaptations pour refléter la diversité de certaines catégories, en particulier la peinture d’histoire, et pour intégrer des genres apparus ultérieurement, comme la peinture abstraite.

Le « motif » désigne quant à lui un élément visuel particulier, un détail significatif ou un groupe d’éléments dans une œuvre, qui peut être isolé ou répété, et parfois associé à d’autres. Il peut être figuratif ou décoratif, notamment dans le cas de motifs ornementaux. 

Enfin, un « élément » est la plus petite unité significative identifiable dans une image. Il peut s’agir d’une figure, d’un détail, d’un objet spécifique, d’un attribut iconographique ou d’une partie d’un motif paysager ou architectural.

Pour éclaircir ces notions, la description sommaire d’un tableau peut servir d’illustration. Prenons l’\textit{Adoration des bergers} peinte à tempera sur bois par le peintre florentin Taddeo Gaddi (vers 1300-1366) vers 1330 et conservé au Musée des Beaux-Arts de Dijon (inv. 1470) (figure \ref{fig:ptrGaddiAdoration}). Elle relève d’un \textit{sujet} biblique, puisque la \textit{scène} représentée provient du Nouveau Testament, plus particulièrement de l’Évangile selon saint Luc (chapitre 2, versets 8-20). Dans l’image, plusieurs \textit{motifs} peuvent être identifiés, comme les bergers, la Vierge tenant l’Enfant Jésus dans ses bras ou encore saint Joseph se reposant en bas à droite de la composition. Le bâton du berger au centre du panneau, l’auréole de la vierge ou la fenêtre visible sur le mur du bâtiment peuvent être désignés comme des \textit{éléments}.

\section[Iconographie et Iconologie]{La lecture de l'image :  Iconographie et Iconologie}

L’analyse des images en histoire de l’art, en particulier par l’identification de leurs thèmes et des scènes représentées, constitue un champ d’étude spécifique, appelé iconographie. La définition et la méthode proposées par l’historien de l’art Erwin Panofsky (1892-1968) dès 1931 sont régulièrement citées pour définir cette discipline\footnote{Erwin Panofsky la définit pour la première fois en 1931 lors d’une conférence à Kiel auprès de la société d’études kantiennes : \cite{panofskyProblemBeschreibungUnd1932}. Il affine ensuite cette définition dans l'introduction de ses \textit{Studies in Iconology} publiés en 1939 puis dans \textit{Meaning in the visual arts} publié en 1955.}. Il définit l’iconographie comme « cette branche de l’histoire de l’art qui se rapporte au sujet ou à la signification des œuvres d’art, par opposition à leur forme »\footcite[p. 19]{panofskyEssaisDiconologieThemes2021}. Plus précisément, le terme iconographie revêt deux sens complémentaires : le premier désigne l’ensemble des représentations d’un même sujet ou thème dans les arts visuels ; le second correspond à la discipline d’histoire de l’art elle-même, centrée sur l’identification, la description et l’interprétation du contenu des images, en examinant les sujets représentés, les compositions, et tous les éléments distincts du style. Ce dernier terme, que Panofsky désigne comme “la forme”, recouvre les qualités plastiques de la représentation, notamment les lignes, les couleurs, la composition, les proportions, les volumes ou encore la lumière. L’iconographie s’intéresse ainsi essentiellement au contenu narratif ou symbolique des œuvres, plutôt qu’à leur aspect formel. 

Le terme d’iconologie est généralement employé en complément de l’iconographie. Les historiens font remonter cette notion, dans son acception proche de son usage contemporain, à Cesare Ripa, érudit italien, qui publie en 1593 un manuel intitulé \textit{Iconologia}\footcite{ripaIconologia1593}, largement diffusé au XVIIᵉ siècle. Cet ouvrage servait de dictionnaire destiné à expliquer la manière de représenter et reconnaître les personnifications et allégories, en précisant leurs attributs\footnote{Voir à ce sujet \cite{okayamaRipaIndexPersonifications1992}}. Comme le souligne Xavier Barral i Altet, le sens du mot iconologie comprenait alors, au moment de la parution de cet ouvrage, ce que nous désignons aujourd’hui par « iconographie »\footcite[p. 24]{barralialtetIntroduction2003}. Au XXᵉ siècle, une nouvelle approche de l’iconologie est développée à Hambourg en Allemagne autour de l’historien d’art Aby Warburg (1866‑1929), puis surtout par Erwin Panofsky, qui entend l’iconologie comme une approche destinée à interpréter le sens profond des images et à dégager leur signification symbolique, en lien avec le contexte culturel, philosophique, religieux ou social de leur création.

Ainsi, Panofsky propose une méthode en trois niveaux pour interpréter le contenu d’une image. Bien que cette démarche ait suscité certaines critiques et ajustements au cours du temps, elle reste aujourd’hui largement respectée pour avoir théorisé la pratique iconographique et iconologique\footcite[en particulier le premier chapitre «What is Iconography?», p. 3-25]{vanstratenIntroductionIconography1994}. Le premier niveau de cette méthode, la description pré-iconographique, consiste à identifier les éléments et les motifs visibles dans la représentation : figures humaines, animaux, objets et leurs relations mutuelles, y compris les événements représentés. Les qualités expressives, telles que la joie, la tristesse ou d’autres intentions, sont également relevées à cette étape. La connaissance de l’histoire du style permet de comprendre comment certains éléments sont représentés dans un contexte particulier : par exemple, un dauphin dans la tradition médiévale est figuré très différemment de la réalité. Le deuxième niveau, l’analyse iconographique, consiste à mettre en relation les motifs et compositions avec des sujets ou des thèmes précis, en identifiant les images, les scènes et les allégories représentées. Le troisième niveau, l’interprétation iconologique, vise à dégager le sens profond de l’image, souvent lié aux choix de l’artiste ou du commanditaire dans la représentation des motifs. Voici un exemple pour illustrer cette démarche : une description pré-iconographique pourrait consister à observer un vieillard levant un couteau et tenant un jeune garçon d’une main, avec un bélier à ses côtés et un ange descendant du ciel. L’analyse iconographique identifierait Abraham sacrifiant son fils, comme raconté au chapitre 22 de la Genèse. L’interprétation iconologique, selon une lecture typologique, pourrait mettre en évidence une allusion à Dieu sacrifiant son fils, préfigurant la Passion du Christ dans le Nouveau Testament. 

Comme le souligne Roelof van Straten, pour l’indexation iconographique, ce sont surtout les deux premières étapes qui importent, c’est-à-dire l’identification du sujet, des scènes, des motifs et des éléments représentés dans l’œuvre\footcite{vanstratenPanofskyICONCLASS1986}. Les conclusions issues d’une interprétation iconologique peuvent également être intégrées à l’indexation, à condition que les concepts soient précis et validés par le contexte de création ou par des sources documentaires. C’est le cas, par exemple, des \textit{Memento mori} présents dans les natures mortes, des représentations de vanités ou des paysages de l’Arcadie. Cela vaut aussi pour les préfigurations dans les associations typologiques entre l’Ancien et le Nouveau Testament.