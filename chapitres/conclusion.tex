L’aboutissement de nos réflexions rejoint celle que Jacques Thuillier formulait déjà il y a quarante ans\footcite{thuillierImageInformatiqueLutilisation1989}. Face à l’essor des premiers outils informatiques dans les institutions patrimoniales, il invitait alors les musées et les bibliothèques non pas à s’en servir pour reconduire des usages préexistants, mais à interroger en profondeur ces pratiques à la lumière des besoins réels des utilisateurs. Le même constat émerge aujourd’hui à travers les expérimentations menées sur l’indexation du RETIF à l’aide des technologies d’intelligence artificielle. Doit-on mobiliser ces outils pour reconduire des pratiques documentaires conçues dans un contexte matériel et scientifique antérieur, ou au contraire les envisager comme une occasion d’élaborer de nouveaux usages, en adéquation avec les possibilités qu’ils offrent et les besoins des utilisateurs ?

Ce projet a notamment mis en évidence plusieurs points d’attention concernant  l’indexation iconographique et les instruments qu’elle emploie, en particulier les vocabulaires contrôlés comme le thésaurus Garnier. Son emploi constitue un véritable défi pour l’automatisation de l’indexation, mais il soulève également des interrogations plus larges sur sa pertinence dans les bases de données iconographiques. Le thésaurus Garnier reste en effet isolé des autres systèmes d’indexation, y compris entre ses propres déclinaisons (POP, AGORHA, Mandragore,...). Depuis sa création dans les années 1980, son usage a été multiple, ses adaptations et modifications institutionnelles compliquant l’interconnexion des bases. Or, depuis plusieurs années, ce vocabulaire semble en perte de vitesse et de plus en plus critiqué par la communauté des historiens de l’art pour son caractère monolingue et sa difficulté à s’adapter aux problématiques contemporaines. Dans un contexte où les bases de données se mondialisent et ne s’adressent plus uniquement à des chercheurs francophones, une enquête approfondie sur son utilisation paraît nécessaire, notamment pour envisager des passerelles avec d’autres systèmes d’indexation.

Un autre problème tient à la difficulté pour les bases de données liées à ce thésaurus de remplir la fonction première de l’indexation : favoriser la recherche documentaire. Les limites d’ergonomie de plateformes comme POP ou AGORHA réduisent fortement l’intérêt de l’indexation iconographique à l’aide du thésaurus Garnier. Ces enjeux d’accessibilité et d’interface devraient être résolus en priorité, avant même d’envisager une automatisation à grande échelle de cette tâche complexe de description des représentations. Ainsi, si le projet d’indexation iconographique assistée par intelligence artificielle du RETIF a permis de révéler ces difficultés, il témoigne aussi d’une certaine tendance à vouloir exploiter des technologies en plein essor, démarche qui gagnerait à s’assortir d’une réflexion préalable quant au véritable intérêt pour les utilisateurs.

Un point crucial demeure donc celui de l’analyse préalable des pratiques de la communauté et de la définition précise des problèmes auxquels l’intelligence artificielle est censée répondre. Cette étape, indispensable pour encourager des méthodes adaptées et guider d’autres projets de ce type, n’a pas pu être menée en profondeur lors des expérimentations autour du RETIF. Une telle enquête aurait pourtant permis de mieux comprendre les attentes réelles de la communauté des usagers et d’orienter le choix de solutions adaptées et pertinentes. Des cadres méthodologiques, tels que ceux proposés par la Bibliothèque du Congrès\footcite{libraryofcongressPlanningFrameworkUsed2025}, ou encore des recommandations issues d’expériences similaires à la nôtre, mériteraient d’être valorisés et diffusés davantage, afin d’inciter les institutions à intégrer l’intelligence artificielle de façon réfléchie et efficace, lorsqu’une approche purement expérimentale n’est pas envisagée.

Un autre enjeu, encore trop peu mis en avant, concerne la qualité des données destinées à constituer des jeux d’entraînement exploitables par les systèmes d’intelligence artificielle. Cette réflexion est aujourd’hui portée par des institutions comme la Bibliothèque nationale de France ou encore ICONCLASS, qui travaillent à améliorer la valeur de leurs corpus et de leur indexation pour nourrir les algorithmes. Elle s’inscrit dans la continuité du rôle des institutions patrimoniales, en France comme à l’étranger, à la fois dans l’inventaire et le catalogage des collections et dans leur mise en libre accès.

Les technologies d’intelligence artificielle apparaissent et évoluent à grands pas, et les possibilités que pourront offrir les outils de vision par ordinateur dans les prochaines années surepasseront très certainement celles d’aujourd’hui. Il est donc préférable de privilégier des chaînes opératoires légères et adaptables, garantissant la pérennité des résultats face à l’émergence de solutions plus performantes. Dans ce contexte, de réelles perspectives se sont révélées grâce à l’usage d’outils comme Panoptic, permettant d’explorer et d’annoter les corpus plus efficacement. Sa capacité à permettre l’exploration de banques d’images et à y associer des métadonnées constitue un atout majeur pour l’indexation iconographique. Si une indexation complète du RETIF devait être entreprise en s’appuyant sur la vision par ordinateur, c’est cette solution que nous préconisons. Elle a l’avantage de prolonger des pratiques établies en histoire de l’art, sans exiger de lourdes manipulations techniques ni de ressources de calcul intensives, et sans nécessiter l’envoi de données vers des serveurs distants pour les effectuer. Elle évite également le risque de produire des informations de qualité hétérogène, ou d’imposer un travail de curation long et fastidieux a posteriori. Nous tenons à souligner ce point : les responsables de collections et les documentalistes n’ont pas la possibilité d’assumer cette charge et la politique de l'institution en matière de curation, notamment le temps consacré ou non à cette pratique, doit être définie dès la conception du projet.

Enfin, l’intelligence artificielle ouvre des perspectives inédites pour l’exploration des corpus iconographiques, notamment grâce aux modèles capables de relier texte et image. Aujourd’hui, la technologie CLIP s’impose comme une solution à la fois performante et efficace, largement adoptée dans de nombreux projets. Ces outils permettent d’interroger les banques d’images non seulement en langage naturel, mais également à partir d’éléments visuels, en exploitant les similarités entre représentations. Cette approche offre de nouvelles possibilités de navigation dans les corpus iconographiques et constitue un atout considérable pour la recherche en histoire de l’art. Elle répond en définitive à l’un des enjeux fondamentaux de l’indexation iconographique depuis ses origines : permettre l’exploration et la découverte des bases de données pour nourrir l’étude et l’investigation scientifiques.

Toutefois, loin d’être écartée au profit des technologies émergentes, l’indexation iconographique doit s’imposer comme un levier essentiel de production de données de qualité, au service à la fois d’analyses iconographiques précises, de méthodes de l’histoire de l’art computationnelle et d’études transdisciplinaires. Son avenir ne réside donc pas dans l’opposition avec les innovations numériques, mais dans leur complémentarité, afin de promouvoir des modèles interopérables et partagés pour la description des représentations.